\chapter*{Statement of Thesis Research}

In the field of robotics, the focus of the research is shifting increasingly from the design and control of single robots towards collaborative methods for multi-robot systems \parencite{8955969}, especially in the collaboration between unmanned aerial and ground vehicles. Aerial vehicles and ground vehicles compensate each other. For example, aerial vehicles extremely expand the vision of ground vehicles. Besides, ground vehicles also extent the battery life of aerial vehicles.

\ac{CAGMRS} finds applications in mapping of environment \parencite{AIRS}, \parencite{RAL_mapping_for_farming}, multi-robot localization \parencite{RA3}, target detection \parencite{TRANS_target_detection} and natural resource monitoring \parencite{3_forest_fire} \parencite{3_concordia_forest_fire}. Multi-Robot collaboration is a typical \ac{CPS}, since it requires high security and real-time response. \ac{CAGMRS} includes three main procedures. The first one, perception, is to perceive the environment and to locate the vehicles via one or multiple sensors, such as mapping and localization. The second one, detection, is to understand the collected information, such as identifying the dangers or targets. The third one, safe control, is to control the vehicles to complete the mission safely, such as path planning and actuators control. The user may control the vehicles remotely, or the command can be made by vehicles themselves. However, in \ac{CAGMRS}, user only interferes with the control when it is necessary.

Therefore, these three procedures of \ac{CAGMRS} will be the fields of focus in this PhD program. There are three recommended journal papers:

\begin{enumerate}
  \item[[ 1]] Yu, X., \& Zhang, Y. (2015). Sense and avoid technologies with applications to unmanned aircraft systems: Review and prospects. Progress in Aerospace Sciences, 74, 152-166.
  \item[[ 2]] Guo, X., Hu, J., Chen, J., Deng, F., \& Lam, T. L. (2021). Semantic Histogram Based Graph Matching for Real-Time Multi-Robot Global Localization in Large Scale Environment. IEEE Robotics and Automation Letters.
  \item[[ 3]] Minaeian, S., Liu, J., \& Son, Y. J. (2015). Vision-based target detection and localization via a team of cooperative UAV and UGVs. IEEE Transactions on Systems, Man, and Cybernetics: Systems, 46(7), 1005-1016.
\end{enumerate} 

The first paper introduces the Sense and Avoid (S\&A) system on \ac{UAVs}, which plays a profoundly important role in integrating UASs into the \ac{NAS} with reliable and safe operations. This paper analyzes the manner of S\&A system, systematically presents an overview on the recent progress in S\&A technologies in the sequence of fundamental functions/components of S\&A in sensing techniques, decision making, path planning, and path following. The approaches to these four aspects are outlined and summarized, based on which the existing challenges and potential solutions are highlighted for facilitating the development of S\&A systems.

The second paper presents a semantic Multi-Robot \ac{SLAM} algorithm, which is to solve the core problem of visual \ac{MR-SLAM}, how to efficiently and accurately perform \ac{MR-GL}. The current methods are highly time-consuming. This paper presents a semantic histogram based graph matching method that is robust to viewpoint variation and can achieve real-time global localization, which is 30 times faster than Random Walk based semantic descriptors. Besides, this paper is open sourced \parencite{website:gxy_mrslam}, which is easier to understand and implement.

The third paper represents a new vision-based target detection and localization system to make use of different capabilities of \ac{AVs} as a cooperative team. The scenario considered in this paper is a team of an UAV and multiple \ac{UGVs} tracking and controlling crowds on a border area. A customized motion detection algorithm is applied to follow the crowd from the moving camera mounted on the UAV. Due to UAV lower resolution and broader detection range, \ac{UGVs} with higher resolution and fidelity are used as the individual human detectors, as well as moving landmarks to localize the detected crowds with unknown independently moving patterns at each time point. The UAV localization algorithm, proposed in this paper, then converts the crowds' image locations into their real-world positions, using perspective transformation. A rule-of-thumb localization method by a UGV is also presented, which estimates the geographic locations of the detected individuals.